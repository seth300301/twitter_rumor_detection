% This must be in the first 5 lines to tell arXiv to use pdfLaTeX, which is strongly recommended.
\pdfoutput=1
% In particular, the hyperref package requires pdfLaTeX in order to break URLs across lines.

\documentclass[11pt, a4paper]{article}

% Remove the "review" option to generate the final version.
\usepackage[review]{acl}

% Standard package includes
\usepackage{times}
\usepackage{latexsym}

% For proper rendering and hyphenation of words containing Latin characters (including in bib files)
\usepackage[T1]{fontenc}
% For Vietnamese characters
% \usepackage[T5]{fontenc}
% See https://www.latex-project.org/help/documentation/encguide.pdf for other character sets

% This assumes your files are encoded as UTF8
\usepackage[utf8]{inputenc}

% This is not strictly necessary, and may be commented out,
% but it will improve the layout of the manuscript,
% and will typically save some space.
\usepackage{microtype}

% If the title and author information does not fit in the area allocated, uncomment the following
%
%\setlength\titlebox{<dim>}
%
% and set <dim> to something 5cm or larger.

\title{COMP90042 Rumour Detection and Analysis on Twitter}

% Author information can be set in various styles:
% For several authors from the same institution:
% \author{Author 1 \and ... \and Author n \\
%         Address line \\ ... \\ Address line}
% if the names do not fit well on one line use
%         Author 1 \\ {\bf Author 2} \\ ... \\ {\bf Author n} \\
% For authors from different institutions:
% \author{Author 1 \\ Address line \\  ... \\ Address line
%         \And  ... \And
%         Author n \\ Address line \\ ... \\ Address line}
% To start a seperate ``row'' of authors use \AND, as in
% \author{Author 1 \\ Address line \\  ... \\ Address line
%         \AND
%         Author 2 \\ Address line \\ ... \\ Address line \And
%         Author 3 \\ Address line \\ ... \\ Address line}

\author{1067992, 1234032, 1237539}

\begin{document}
\maketitle

\section{Introduction}

With the popularity of the internet and social media in recent years, social media has become an indispensable part of people's lives. More and more people are using social media not only to connect with family and friends, but also to keep up to date with what is currently happening. The role of the user has also changed from that of a receiver of information to that of a creator. Many news outlets publish news on social media, and ordinary users can also express their views and opinions. However, the growth of social media has accelerated the spread of information and has also brought about a proliferation of false rumours. Twitter, one of the most popular social media, has generated a large number of rumours that are difficult to verify and are spreading rapidly among users. Therefore, it is crucial to identify whether a message is a rumour or not.

\section{Related Works}

The topic of rumour identification and analysis has attracted a great deal of attention in recent years. 

\citet{tian20202} transformed the rumour detection task into a supervised classification problem and trained the model using a set of labelled source tweets and comments. They found that comments responded to different user attitudes towards rumours and non-rumours, and based on this decomposed the problem into rumour detection for detecting positions and tweets from comments. Convolutional neural networks (CNN) and BERT neural network language models were used for the experiments. The article further proposes a rumour detection model based on a combination of these two models with significant advantages for early rumour detection.

Some studies analysed whether a tweet is a rumour by using features other than the content. \citet{kwon20132} detected rumours by analysing time, structure and language. They found that non-rumours usually had only one significant spike, while rumours tended to have multiple periodic spikes. They also found that rumours and non-rumours differed significantly in the user diffusion network. A new time series fitting model and network structure are used to train the data. This article shows a more accurate result of identifying rumors from other type of information.

\citet{wu20152} examined the problem of false rumours on the popular Chinese social network Weibo. In addition to studying traditional semantic features such as topic-based and sentiment-based features, propagation patterns were also investigated through a hybrid Support Vector Machine (SVM) classifier based on a graph-kernel. They found that the false rumor is first posted by a normal user, then reposted and supported by some opinion leaders and finally reposted by a large number of normal users. On the contrary, the normal message is posted by an opinion leader and reposted directly by many normal users. Based on these findings, the model can be used to detect false rumours early, and 90\% of rumours can be detected just one day after their initial broadcast.

\section{Dataset}

This work trains a model based on given training data, optimises the model based on development data, tests the model on test data and ultimately applies it to determine whether tweets related to covid are rumours. The training dataset, the development dataset and the covid dataset are all tweet IDs of source tweets and replies, corresponding to the two sets of tags in separate files. When crawling tweets by tweet IDs, some tweets cannot be retrieved because they have been deleted by the users, so these tweets will be ignored from the data.

Before building the rumour analysis model, the tweet data was analysed. 

\begin{table}[]
    \centering
    \begin{tabular}{lc}
        \hline
        \textbf{Dataset} & \textbf{Percentage}\\
        \hline
        \verb|train-rumour| & 0.2216 \\
        \verb|train-nonrumour| & 0.7784 \\
        \verb|dev-rumour| & 0.2199 \\
        \verb|dev-nonrumour| & 0.7801 \\
        \hline
    \end{tabular}
    \caption{Share of rumours and non-rumours in the dataset}
    \label{tab:my_label}
\end{table}

\begin{table}[]
    \centering
    \begin{tabular}{ccc}
        \hline
        \textbf{Dataset} & \textbf{Median} & \textbf{Mean}\\
        \hline
        \verb|train-rumour| & 14 & 30.2 \\
        \verb|train-nonrumour| & 5 & 10.7 \\
        \verb|dev-rumour| & 12 & 26.8 \\
        \verb|dev-nonrumour| & 5 & 12.6 \\
        \hline
    \end{tabular}
    \caption{Number of replies}
    \label{tab:my_label}
\end{table}

\section{Task 1}

\subsection{detection system}

introduce our detection system
the reason behind the choices

\subsection{performance}

\section{Task 2}

\subsection{topics}
What are the topics of COVID-19 rumours, and how do they differ from the non-rumours?

How do COVID-19 rumour topics or trends evolve over time?

\subsection{hashtags}
What are the popular hashtags of COVID-19 rumours and non-rumours? How much overlap or difference do they share?

\subsection{sentiment}
Do rumour source tweets convey a different sentiment/emotion to the non-rumour source tweets? What about their replies?

\subsection{users}
What are the characteristics of rumour-creating users, and are they different to normal users?

% Entries for the entire Anthology, followed by custom entries
\bibliography{anthology,custom}
\bibliographystyle{acl_natbib}

\appendix

\section{Example Appendix}
\label{sec:appendix}

This is an appendix.

\end{document}
